It can be seen that the absorption of $\beta$-particles is only roughly exponential. That arises from the fact that $\beta$-particles are emitted from a radioisotope with a range of kinetic energies. Thus absorption coefficients determined from Beer's law are only approximate.

The more penetrating component is due to \textbf{Bremsstrahlung}, electromagnetic radiation resulting from the rapid acceleration and deceleration of $\beta$-particles travelling through the material. Being electromagnetic in character it is less easily absorbed and hence more penetrating than the original $\beta$-particles.

The end-point energy of the $\beta$-particles can be estimated from their range $R$ in a particular absorber. $R$ is defined as the total thickness ($\rho \, x$) of absorber through which a $\beta$-particle of maximum energy will traverse before coming to rest. The end-point energy is related to the range by the empirical Feather relationship:
\begin{displaymath}
E = 1.85R + 0.245
\end{displaymath}
where $E$ is in MeV and $R$ in g cm$^{-2}$.