\subsection{Uncertainties}

Radioactivity is a random process in the sense that it is not possible to predict when the next decay will occur. However, radioactivity has a mean rate of decay, which depends on the source atom. This mean rate will imply a mean rate of detections (counts) per unit time.

In this second experiment, we would like to measure how well can we estimate the mean count rate, and how our precision depends on the number of counts we accumulate.

Place the source at a distance from the Geiger counter such that you record approximately 5 counts in 10 seconds. As you have seen in \S \ref{distance} the count rate depends on the separation between the source and the Geiger counter, so it is important to keep a constant distance between when measuring the count rate.

Now record the number of counts detected in 10 seconds. Write down the value and repeat the experiment 30 times.