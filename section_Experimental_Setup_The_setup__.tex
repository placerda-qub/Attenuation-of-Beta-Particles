\section{Experimental Setup}

The setup consists of a radioactive source and a Geiger counter, mounted on separate holders. Particles originating at the source travel through air and other materials we may place in their path, and are individually detected at the Geiger counter.

When no measurement is being taken, the radioactive source should be kept in its shield.

The materials that will be used to attenuate the $\beta$ particles are discs of aluminium of three different thicknesses: 0.01, 0.1, and 1 mm. These can be stacked to produce intermediate thicknesses. The discs are placed inside a cylindrical end-piece, on top of an existing metallic O-ring. After the discs are in place, a hollow cylinder is inserted into the end-piece, holding everything in place. Finally, when ready to perform a measurement, the radioactive source is inserted into the other end of the hollow cylinder, and the entire setup is clipped horizontally onto the holder.

To perform a measurement, align the source setup with the Geiger counter setup as well as possible.