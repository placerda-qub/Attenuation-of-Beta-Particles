\section{Background}% \\ 
%{\small Deadline: 2016/09/28 at 2 pm}}

In this experiment we will study the attenuation of beta particles in aluminium, using a radioactive source and a Geiger counter detector. 

The source used is strontium-$90$ ($^{90}$Sr) which has an activity of $5\,\mu$Ci\footnote{The curie (Ci) is an old-fashioned but convenient measure of radioactivity and indicates the number of decays per unit time. The more modern SI unit of radioactivity is the becquerel (Bq), which corresponds to $1$ decay per second. The conversion between the two units is $1\,\mu\textrm{Ci} = 37,000\,\textrm{Bq} = 37\,\textrm{kBq}$.} Strontium-$90$ undergoes $\beta^{-}$ decay into yttrium-$90$ with decay energy $0.546$ MeV and half-life $28.79$ years. 

Each decay emits an electron ($\mathrm{e}^-$)  and an electron anti-neutrino ($\overbar{\nu_\mathrm{e}}$). Neutrinos have negligible probability of interacting with matter. Electrons, however, lose kinetic energy as they interact electromagnetically with matter. So, they can only travel a certain distance, or range, before they come to rest.

In this experiment we will place pieces of aluminium between the source and the detector to estimate the range of electrons.  The electrons leave the source at a given rate, $I_0$, measured in counts/second. As they traverse a thickness $x$ of aluminium, only an attenuated fraction $I(x)$ survives absorption. The surviving fraction is given by:

\begin{displaymath}
I=I_0\,\exp\left(-\mu\,x\right)
\end{displaymath}

where $\mu$ is the attenuation coefficient.

The electrons leave 

