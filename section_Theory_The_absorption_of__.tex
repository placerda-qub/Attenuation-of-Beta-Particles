\section{Theory}

The absorption of $\beta$ rays is given approximately by \textbf{Beer's Law}
\begin{displaymath}
I_x = I_0 \, \exp\left(\mu \, x\right)
\end{displaymath}

where:
\begin{itemize}
\item $I_x$ is the intensity after passing through an absorber of thickness $x$;
\item $I_0$ is the initial intensity at the source;
\item $\mu$ is the linear absorption coefficient.
\end{itemize}

Thus, a plot of $\ln I_x$ against $x$ theoretically results in a straight line whose negative slope gives $\mu$.

In practice the linear absorption coefficient $\mu$ will vary widely from one absorber material to another. However, the quantity $\mu/\rho$ is approximately constant for all types of absorbers and hence the use of this quantity, the mass absorption coefficient ($\mu_m$) is to be preferred. This necessitates expressing the thickness of absorber not in units of length, but in g cm$^{-2}$. Beer's exponential absorptions law becomes
\begin{displaymath}
I_x = I_0 \, \exp \left[-(-\mu/\rho)(\rho x)\right]
\end{displaymath}

where $\rho$ is the density of absorber, and 
\begin{displaymath}
\ln I_x = \ln I_0 - (\mu_m) (\rho x)
\end{displaymath}

Hence a plot of $\ln I_x$ against thickness, expressed as $(\rho x)$ results in a straight line whose slope gives the mass absorption coefficient ($\mu_m$).

A typical absorption curve is illustrated in Figure 1. It shows two main regions of interest - the sloping beta absorption component, A, and a more penetrating component, B.