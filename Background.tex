\section{Background}% \\ 
%{\small Deadline: 2016/09/28 at 2 pm}}

In this experiment we will study the attenuation of beta particles in aluminium, using a radioactive source and a Geiger counter detector. 

The source used is strontium-$90$ which undergoes $\beta^{-}$ decay into yttrium-$90$ with decay energy $0.546$ MeV and half-life $28.79$ years. 

Each decay emits an electron ($\mathrm{e}^-$)  and an electron anti-neutrino ($\overbar{\nu_\mathrm{e}}$). Neutrinos have negligible probability of interacting with matter. Electrons, however, lose kinetic energy as they interact electromagnetically with matter. So, they can only travel a certain distance, or range, before they come to rest.